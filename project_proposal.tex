% Options for packages loaded elsewhere
\PassOptionsToPackage{unicode}{hyperref}
\PassOptionsToPackage{hyphens}{url}
%
\documentclass[
]{article}
\usepackage{amsmath,amssymb}
\usepackage{iftex}
\ifPDFTeX
  \usepackage[T1]{fontenc}
  \usepackage[utf8]{inputenc}
  \usepackage{textcomp} % provide euro and other symbols
\else % if luatex or xetex
  \usepackage{unicode-math} % this also loads fontspec
  \defaultfontfeatures{Scale=MatchLowercase}
  \defaultfontfeatures[\rmfamily]{Ligatures=TeX,Scale=1}
\fi
\usepackage{lmodern}
\ifPDFTeX\else
  % xetex/luatex font selection
\fi
% Use upquote if available, for straight quotes in verbatim environments
\IfFileExists{upquote.sty}{\usepackage{upquote}}{}
\IfFileExists{microtype.sty}{% use microtype if available
  \usepackage[]{microtype}
  \UseMicrotypeSet[protrusion]{basicmath} % disable protrusion for tt fonts
}{}
\makeatletter
\@ifundefined{KOMAClassName}{% if non-KOMA class
  \IfFileExists{parskip.sty}{%
    \usepackage{parskip}
  }{% else
    \setlength{\parindent}{0pt}
    \setlength{\parskip}{6pt plus 2pt minus 1pt}}
}{% if KOMA class
  \KOMAoptions{parskip=half}}
\makeatother
\usepackage{xcolor}
\usepackage[margin=1in]{geometry}
\usepackage{graphicx}
\makeatletter
\def\maxwidth{\ifdim\Gin@nat@width>\linewidth\linewidth\else\Gin@nat@width\fi}
\def\maxheight{\ifdim\Gin@nat@height>\textheight\textheight\else\Gin@nat@height\fi}
\makeatother
% Scale images if necessary, so that they will not overflow the page
% margins by default, and it is still possible to overwrite the defaults
% using explicit options in \includegraphics[width, height, ...]{}
\setkeys{Gin}{width=\maxwidth,height=\maxheight,keepaspectratio}
% Set default figure placement to htbp
\makeatletter
\def\fps@figure{htbp}
\makeatother
\setlength{\emergencystretch}{3em} % prevent overfull lines
\providecommand{\tightlist}{%
  \setlength{\itemsep}{0pt}\setlength{\parskip}{0pt}}
\setcounter{secnumdepth}{-\maxdimen} % remove section numbering
\ifLuaTeX
  \usepackage{selnolig}  % disable illegal ligatures
\fi
\usepackage{bookmark}
\IfFileExists{xurl.sty}{\usepackage{xurl}}{} % add URL line breaks if available
\urlstyle{same}
\hypersetup{
  pdftitle={Team Registration and Project Proposal},
  hidelinks,
  pdfcreator={LaTeX via pandoc}}

\title{Team Registration and Project Proposal}
\author{}
\date{\vspace{-2.5em}2024-11-07}

\begin{document}
\maketitle

\textbf{Group members (name and uni)}: Pradeeti Mainali (pm3260), Sining
Leng (sl5454), Polly Wu(rw3031), Shizhe Zhang (sz3214), Yan Li(yl5505)

\textbf{Tentative Project Title}: No Zombie Apocolypse Here: Exploring
Mushrooms

\textbf{Motivation for this project}: With an increase in people who
forage for mushrooms, we want to explore this data set to see if there
are any striking characteristics of mushrooms that can be used to
determine if they are poisonous or edible.

\textbf{Intended final products}:

\begin{itemize}
\tightlist
\item
  Website describing the data set and our regression results; a screen
  cast of our website.
\item
  A written report that details how we complete our project.
\end{itemize}

\textbf{Anticipated data sources}:
\href{https://archive.ics.uci.edu/dataset/73/mushroom}{Mushroom dataset
from UC Irvine}

This data set includes descriptions of hypothetical samples
corresponding to 23 species of gilled mushrooms in the Agaricus and
Lepiota Family (pp.~500-525). Each species is identified as definitely
edible, definitely poisonous, or of unknown edibility and not
recommended.

\textbf{Planned analyses/visualizations/coding challenges}:

\begin{itemize}
\tightlist
\item
  \emph{Planned Analyses}:

  \begin{itemize}
  \tightlist
  \item
    Baseline characteristics comparison for mushrooms categorized as
    poisonous or edible
  \item
    Univariate regression for how each feature (cap\_shape,
    cap\_surface, cap\_color, bruises, etc. ) is associated with
    poisonous
  \item
    Multivariate regression model for predicting whether a mushroom is
    poisonous
  \end{itemize}
\item
  \emph{Visualizations}: Hexbin plot to show the number distribution of
  classes/habitats v.s. color (cap/stalk/veil/spore).
\item
  \emph{Coding challenges}: picking co-variates for the multivariate
  model
\end{itemize}

\textbf{Planned timeline}:

\begin{itemize}
\tightlist
\item
  November 8th: Submit the Proposal
\item
  November 11-15: Project review meeting
\item
  Nov 17: Finalizing plans and starting work (Internal deadline).
\item
  Nov 24: Webpage work due (internal deadline)
\item
  Dec 1: Screencast recoded and report written (internal deadline). The
  final review of work starts.
\item
  December 7: Report, webpage, and screencast, peer assessment due.
\item
  December 12th: In-class discussion of projects
\end{itemize}

\end{document}
